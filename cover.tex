% -*-latex-*-

% NOTE:

%% By default, the thesis will be copyrighted to MIT.  If you need to copyright
%% the thesis to yourself, just specify the `vi' documentclass option.  If for
%% some reason you want to exactly specify the copyright notice text, you can
%% use the \copyrightnoticetext command.  
%\copyrightnoticetext{\copyright IBM, 1990.  Do not open till Xmas.}

% If there is more than one supervisor, use the \supervisor command
% once for each.
%\supervisor{William J. Supervisor}{Associate Professor}

% This is the department committee chairman, not the thesis committee
% chairman.  You should replace this with your Department's Committee
% Chairman.
%\chairman{Arthur C. Chairman}{Chairman, Department Committee on Graduate Theses}

% Make the titlepage based on the above information.  If you need
% something special and can't use the standard form, you can specify
% the exact text of the titlepage yourself.  Put it in a titlepage
% environment and leave blank lines where you want vertical space.
% The spaces will be adjusted to fill the entire page.  The dotted
% lines for the signatures are made with the \signature command.
\maketitle

% The abstractpage environment sets up everything on the page except
% the text itself.  The title and other header material are put at the
% top of the page, and the supervisors are listed at the bottom.  A
% new page is begun both before and after.  Of course, an abstract may
% be more than one page itself.  If you need more control over the
% format of the page, you can use the abstract environment, which puts
% the word "Abstract" at the beginning and single spaces its text.

%% You can either \input (*not* \include) your abstract file, or you can put
%% the text of the abstract directly between the \begin{abstractpage} and
%% \end{abstractpage} commands.
\cleardoublepage

\section*{CERTIFICATE OF ORIGINALITY}

 \label{sec:certificate}
  \begin{center}
  \end{center}
	%\vspace*{\fill}
  \vspace{1.0cm}
  \noindent
I hereby declare that this thesis is my own work and that, to the best of
my knowledge and belief, it reproduces no material previously
published or written, nor material that has been accepted for the award
of any other degree or diploma, except where due acknowledgement has
been made in the text.
  \par\noindent
  \vspace{\stretch{0.1}}
  \begin{flushleft}
    \begin{tabular}{l p{0.5in} p{1in} p{0.5in} l}
      & & & & (Signed) \\
      \cline{1-4}\\
      & & {Zhenlin An} & & (Name of student) \\
      \cline{1-4}\\
    \end{tabular}
  \end{flushleft}
  \vspace{\fill}


% First copy: start a new page, and save the page number.
\cleardoublepage
% Uncomment the next line if you do NOT want a page number on your
% abstract and acknowledgments pages.
% \pagestyle{empty}
\setcounter{savepage}{\thepage}
\begin{abstractpage}
\input{abstract}
\end{abstractpage}

% Additional copy: start a new page, and reset the page number.  This way,
% the second copy of the abstract is not counted as separate pages.
% Uncomment the next 6 lines if you need two copies of the abstract
% page.
% \setcounter{page}{\thesavepage}
% \begin{abstractpage}
% \input{abstract}
% \end{abstractpage}

\cleardoublepage

\section*{Acknowledgments}

This is the acknowledgements section. You should replace this with your
own acknowledgements.

\cleardoublepage

\section*{Previous Published Material}

This is the published section. You should replace this with your
own published material.

%%%%%%%%%%%%%%%%%%%%%%%%%%%%%%%%%%%%%%%%%%%%%%%%%%%%%%%%%%%%%%%%%%%%%%
% -*-latex-*-
